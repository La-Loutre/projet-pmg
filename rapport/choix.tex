\section {Choix effectués}

\subsubsection{Algorithme de référence}

Un premier algorithme naïf a été écrit. Il s'agit d'une simple double
boucles qui, pour chaque case, fait écrouler le tas de sable sur les
quatres cases voisines.  Nous utilisons cette algorithme comme
référence pour vérifier que les algorithmes optimisés ou parallèles
donnent des résultats corrects. Il s'agit de la méthode
\texttt{compute\_naive} (algorithme~).

% TODO: comment cest alloc memoire

\subsection{Algorithmes séquentiels}

\subsubsection{Division euclidienne}

Une première optimisation simple consiste à effectuer des divisions
euclidiennes pour faire ébouler un tas de sable d'un seul coup sur les
cases voisines. L'avantage est que si un case contient un grand nombre
de grains de sable, on va pouvoir la vider en une
opération. Seulement, à l'itération suivante, un huitième des grains
de sable envoyés chez les voisins reviennent sur la case de départ.
Il s'agit de la méthode \texttt{compute\_eucl} (algorithme~).
\medskip

\subsubsection{Prédiction de branchement}

Une autre optimisation consiste à mieux utiliser la prédiction de
branchement du processeur.  À chaque itération et pour chaque case, on
vérifie si la case dépasse une certaine valeur (la taille maximale
d'un tas de sable avant éboulement). Or, comme d'une case à l'autre le
contenu peut être radicalement différent, on peut peut pas
statistiquement prédire si nous allons effectuer un changement (un
éboulement) ou non. La prédiction de branchement n'est donc pas
efficace ici.
\medskip

Il faudrait alors effectuer toujours les mêmes opérations, qu'il y ait
un changement ou non. En réalité, cela est compliqué.
% TODO: ici c'est pas parfait ya un vieux switch
\medskip

\subsubsection{Diminuer les accès en écriture}

Une autre approche que l'éboulement du tas de sable sur les cases
voisines consistes à consulter le hauteur du tas de chacun de des
voisins d'une case. En fonction de ces quatres hauteurs, on met à jour
la valeur de la case, en faisant écrouler les tas de sables voisins
sur la case, mais sans modifier le contenu des cases voisines. Avec
cette approche, à chaque itération et pour chaque case, on effectue 5
accès lecture et un accès écriture contre 5 accès lecture et écriture
pour l'autre approche. Il s'agit de la méthode
\texttt{compute\_eucl\_swap} (algorithme~TODO).

L'incovenient de cette approche est qu'on ne peut pas lire et écrire
dans la même matrice sinon les résultats seraient faux. Ainsi, on
dispose d'une matrice d'écriture et d'une matrice de lecture. Pendant
qu'on lit la matrice de lecture, on écrit sur une matrice
auxiliaire. Ensuite, après une itération, on échange la matrice de
lecture et d'écriture.

Nous pouvons avec cette méthode facilement effectuer le même
traitement qu'il faille ébouler une case ou non.
\bigskip

Nous avons essayé d'autres approches comme la détection des zones
stables, \texttt{compute\_eucl\_chunk}, ou l'utilisation d'opérations vectorielles pour tenter de
traiter plusieurs cases d'un seul coup, \texttt{compute\_eucl\_vector}.

\subsection{Algorithmes multi-threads}

\subsubsection{Paralléliser une itération de l'algorithme}

Une première solution multi-threads a été réalisée à l'aide
d'OpenMP. Il s'agit de la méthode \texttt{compute\_omp}.  Nous
répartissons le traitement de chaque ligne de la matrice sur un
ensemble de threads. Un thread se voit donc attribuer
$(DIM-2)/nthreads$ lignes successives. Comme pour
\texttt{compute\_eucl\_swap}, lorsqu'on parcourt une case, on ne
modifie pas les cases voisines. Cela permet de ne pas modifier les
parties de la matrices gérées par d'autres threads.

Chaque thread travaille donc sur une matrice privée de taille
$DIM*DIM$ dans la pile. Une première itération de l'algorithme est
faite, chaque thread modifie sa propre matrice, puis on synchronise
tous les threads avec une barrière. Cela nous permet de mettre à jour
la matrice de départ avec les nouvelles valeurs suite au débordement
de chaque case.
\medskip

\subsubsection{Paralléliser p-itération de l'algorithme}

% TODO: parler de p iter

\subsubsection{Rapatriement des données}

Une autre approche a été pensée pour éviter le rapatriement des
nouvelles valeurs sur la matrice de départ. Celle-ci se base plus
sur \texttt{compute\_eucl\_swap}. Chaque thread ne travaille plus sur
une matrice privée mais on dispose de deux matrices partagées, une en
lecture et une en écriture.

Chaque thread effectue une itération en lisant dans une matrice et en
écrivant dans l'autre. Lorsque tous les threads ont terminés, on
échange la matrice de lecture et d'écriture.

Il est nécessaire de synchroniser les threads à chaque itération pour
être certain que tout le monde s'arrête lorsque toute la matrice est
stabilisée. Il s'agit de la méthode \texttt{compute\_omp\_swap} (algorithme~).
\medskip

\subsubsection{Limiter au maximum les synchronisations}

Une autre approche a été tentée, avec l'objectif d'utiliser des
barrières le moins souvent possible. Il s'agit de la méthode
\texttt{compute\_omp\_swap\_nowait} (algorithme~). Cette méthode prend
le pari de les threads vont évoluer à la même vitesse.

Ainsi, chaque thread tente de stabiliser la région qui lui est
assignée sans communication avec les autres threads. Il est important
de rappeler comment est réparti le travail. Chaque thread s'occupe de
$(DIM-2)/nthreads$ lignes successives de la matrice. Pour garder des
résultats corrects, il faut insérer des dépendances sur les lignes
frontières de deux threads. Ces dépendances sont illustrées sur la
figure~TODO.

Pour implémenter ces dépendances, nous utilisons un tableau de
sémaphores de taille $nthreads-1$ et initialisées à 0. Chaque thread
$t$ (excepté le dernier thread) tente de décrémenter la sémaphore du
thread $t+1$ avant d'écrire la dernière ligne qui lui est attribuée,
la ligne frontière entre deux threads. De son côté, le thread $t+1$
incrémente la sémaphore lorsqu'il a terminé de traiter la première
ligne qui lui est attribuée.

On espère en pratique que les threads travaillent à la même vitesse
et que la sémaphore contient toujours la valeur 1.

Lorsqu'un thread a stabilisé sa partie de la matrice, cela ne veut pas
dire qu'elle va rester stable, jusqu'à la fin. Les éboulement peuvent
ne pas encore être parvenu à ses frontières comme cela peut être le
cas lorsqu'on initialise tous les cases à 0 sauf celle du milieu où on
place 100000. Ainsi, chaque thread ayant stabilisé sa matrice s'endort sur
une condition (une autre sémaphore).

Pour le le dernier thread a avoir stabilisé sa matrice, plutôt que de
s'endormir lui aussi, il réveille tous les autres en leur demandant
d'effectuer une dernière itération. Cette itération doit permettre de
savoir si des nouvelles valeurs ont été placées aux frontières pendant
qu'on était endormis. Si c'est le cas, on recommence le processus
depuis le départ. Sinon, l'algorithme se termine.

\subsubsection{Version OpenCL}
