\section{Perspectives}

\subsection{Réduction du champ de recherche}

Plutôt que de détecter les zones stables, on pourrait détecter
\textit{la} zone instable de la matrice.

Pour réduire le temps que prend une itération, nous pouvons essayer de
détecter le plus petit rectangle de la matrice qui contient toutes les
cases instables.

Il faut donc garder 4 entiers pour garder en mémoire la colonne
minimal et maximal ainsi que la ligne minimal et maximal entre
lesquelles se trouvent l'intégralité des cases instables.

Cette méthode devrait permettre de réduire le temps d'exécution du
programme pour le cas de la tour de sable.

En revanche, cela n'apporte rien en terme de parallélisation.

\subsection{Suppression des barrières}

Notre méthode \texttt{compute\_omp\_swap\_nowait} n'a pas pu être
terminée à temps. Il aurait été intéressant de vérifier dans la
pratique si prendre le pari que les threads travaillent à la même
vitesse (à une itération près) est gagnant ou pas.

En effet, on ne perd presque pas de temps à appeler la fonction
\texttt{sem\_wait()} lorsque la sémaphore contient la valeur 1, et
chaque thread effectue cette vérification que une ou deux fois par
itération.

En revanche, il se peut que la probabilité que la sémaphore contienne
0 à l'instant où un thread souhaite traiter une frontière soit
suffisamment trop élevée pour que cette méthode soit intéressante (à cause des dépendances ``arrières'').
